\section{Related Work}
\label{sec:related_work}
\subsection{Faulty code detection}
Faulty code defection is a significant research field in software engineering \cite{minku2016data}. The  defect  prediction  literature can be divided into two approaches: First,  by  using Software Metrics, the measurable properties of the software system and second, by using fault data from a similar code snippet. In the first approach, authors focus on designing new discriminative features which can distinguish  between  types  of  faulty  code. Authors in \cite{nagappan2007using} proposed churn metrics and combined it with software dependencies for defect prediction. The efficiency of change metrics and static code attributes for defect prediction are analysed comprehensively in \cite{moser2008comparative}. To select appropriate features, authors in \cite{arar2017feature} proposed to use Naive Bayesian method to filter redundant ones. These approaches is strongly depended on the designing metrics which may be redundant or not be highly correlated with class labels. Additionally, hand-crafted metrics cannot make full use of code context information to mine the syntactic structure and semantic information of code snippets. To overcome this obstacle, AST and CFG can be used to represent code snippets to preserve the syntactic structure and semantic information of code. Recently, machine  learning  techniques  are  being  used  in  software  fault  prediction to assist testing and maintainability. In the literature, a machine learning model have been constructed in \cite{kumar2018effective} which focused on Least square support vector machine with linear, polynomial and radial basis kernel functions. In \cite{wang2016automatically}, DBN is employed to generate hidden features, which contains syntaxes and semantics of programs and a classifiers is used to predict the faulty code by training on these hidden feature. Long short-term memory (LSTM) network is used in \cite{lin2018cross} to learn the representation of programs' ASTs to discover faulty code. \cite{li2017software} proposed a hybrid model between software metrics and the features learned by convolutional neural network (CNN). A novel graph convolutional network is designed in \cite{phan2017convolutional} to learn the semantic features of source code by learning on the program' CFGs. Applying deep learning models can automatically extract the discriminative features however, its drawback is that it requires a large amount of data while it is difficult in FCD problem. In the literature, few studies have focused on solving this issue. 

\subsection{Data limitation and Imbalance data}
Deep artificial neural networks require a large corpus of training data in order to effectively learn. Limited training data results in a poor approximation. An over-constrained model will underfit the small training dataset, whereas an under-constrained model, in turn, will likely overfit the training data, both resulting in poor performance.
Transfer learning and data augmentation are two commonly used techniques for dealing with data limitation. Transfer learning is used to improve a classifier from one domain by transferring information from a related domain \cite{weiss2016survey}. There are many fields that transfer learning has been successfully applied to including computer vision \cite{wang2011heterogeneous}, \cite{zhu2011heterogeneous} and nature language processing with BERT \cite{devlin2018bert}, ULMFiT \cite{howard2018universal}, ElMo \cite{peters2018deep} and GPT \cite{radford2019language}.
Data   augmentation help to increase  the  amount of training data using information only in training data. In \cite{bae2018perlin}, authors present a general data augmentation strategy using Perlin noise, applying it to pixel-by-pixel image classification and quantification of various kinds of image patterns of diffuse interstitial lung disease (DILD). Authors in \cite{perez2017effectiveness} propose a method to allow a neural net to learn augmentations that best improve the classifier, which authors call neural augmentation which proven the success on various datasets.
In FCD problem, authors in \cite{nam2017heterogeneous} and \cite{viet2019transfer} proposed to use transfer learning technique to deal with the data limitation issue.


Our proposed deep learning network differs from the aforementioned faulty code detection methods. It focus on dealing with data limitation in FCD problem by using a novel model architecture and new training strategy.